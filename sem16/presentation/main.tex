\documentclass[10pt,pdf,hyperref={unicode}]{beamer}

\usepackage[normalem]{ulem}
\usepackage{qrcode}
\usepackage{array}
\usepackage[T2A]{fontenc}
\usepackage[utf8]{inputenc}
\usepackage{colortbl}
\usepackage{minted}
\usepackage{listings}
\usepackage{tcolorbox}

\setbeamertemplate{navigation symbols}{}

\usetheme{default}

\usepackage{array}
\newcolumntype{L}[1]{>{\raggedright\let\newline\\\arraybackslash\hspace{0pt}}m{#1}}
\newcolumntype{C}[1]{>{\centering\let\newline\\\arraybackslash\hspace{0pt}}m{#1}}
\newcolumntype{R}[1]{>{\raggedleft\let\newline\\\arraybackslash\hspace{0pt}}m{#1}}

\definecolor{shadecolor}{RGB}{210,210,210}
\newcommand{\asm}[1]{\mintinline{gas}{#1}}

\newcommand{\qrlinkframe}[2]{\begin{frame}{#1}
\center\qrcode[hyperlink,height=75px]{#2}
\end{frame}}

\title{Семинар 16: управление процессами}
\date{9 апреля, 2020}


\begin{document}

\begin{frame}
  \titlepage
\end{frame}

\begin{frame}
\center\HugeАкт 1: mini GDB
\end{frame}

\begin{frame}{ptrace}
\begin{itemize}
    \item ptrace — интерфейс ядра, предоставляющий возможность наблюдать и контролировать исполнение другого процесса
    \item tracer — это тот, \emph{кто} контролирует выполнение
    \item tracee — это тот, \emph{кого} контролируют
\end{itemize}
\end{frame}

\begin{frame}[fragile]
\frametitle{ptrace: attaching}
\begin{center}
\begin{minipage}{0.95\textwidth}
\begin{minted}{c}

pid_t pid = fork();
if (pid == 0) {
    ptrace(PTRACE_TRACEME, 0, NULL, NULL);
    execve(...);
} else {
    for (;;) {
        int status = 0;
        waitpid(pid, &status, 0);
        if (WIFSTOPPED(status)) {
            // процесс остановился, из-за ptrace
        } else if (WIFSIGNALED(status)) {
            // процесс был просигнализирован
        } else if (WIFEXITED(status)) {
            // процесс завершился
        } else {
            // никогда не выполняется
        }
    }
}

\end{minted}
\end{minipage}
\end{center}
\end{frame}

\begin{frame}{Как ставятся брейпоинты?}
\begin{itemize}
    \item В Linux поддерживается соглашение, что \asm{int 3} будет вызывать \textbf{SIGTRAP}
    \item Команды PTRACE\_PEEKUSER и PTRACE\_POKEUSER используются, чтобы читать/писать машинные слова в child-процессе
    \item Идея — давайте вместо оригинальной инструкции запишем \asm{int 3}, когда процесс остановится, посмотрим где остановились, восстановим инструкцию и повторим её
\end{itemize}
\end{frame}


\begin{frame}{Как повторить инструкцию?}
\begin{itemize}
    \item В прерываниях процессора это делается автоматически (на самом деле, зависит от типа прерывания)
    \item Нам придётся вручную менять RIP для этого
    \item С помощью PTRACE\_GETREGS и PTRACE\_SETREGS можно изменять регистры
\end{itemize}
\end{frame}


\begin{frame}[fragile]
\frametitle{ptrace: struct pt\_regs}
\begin{center}
\begin{minipage}{0.95\textwidth}
\begin{minted}{c}

struct pt_regs {
    unsigned long r15;
    unsigned long r14;
    unsigned long r13;
    unsigned long r12;

    /* ... */

    unsigned long rax;
    unsigned long orig_rax;

    /* ... */

    unsigned long rip;
    unsigned long cs;
    unsigned long eflags;
    unsigned long rbp;
    unsigned long rsp;
    unsigned long ss;
};

\end{minted}
\end{minipage}
\end{center}
\end{frame}


\begin{frame}
\center\HugeАкт 2: mini docker
\end{frame}


% \begin{frame}
% \begin{itemize}
% \end{itemize}
% \end{frame}

\begin{frame}
\center\Huge{Gratias ago!}
\end{frame}


\end{document}
