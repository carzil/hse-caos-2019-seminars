\documentclass[10pt,pdf,hyperref={unicode}]{beamer}

\usepackage[normalem]{ulem}
\usepackage{qrcode}
\usepackage{array}
\usepackage[T2A]{fontenc}
\usepackage[utf8]{inputenc}
\usepackage{colortbl}
\usepackage{minted}
\usepackage{listings}
\usepackage{tcolorbox}

\setbeamertemplate{navigation symbols}{}

\usetheme{default}

\usepackage{array}
\newcolumntype{L}[1]{>{\raggedright\let\newline\\\arraybackslash\hspace{0pt}}m{#1}}
\newcolumntype{C}[1]{>{\centering\let\newline\\\arraybackslash\hspace{0pt}}m{#1}}
\newcolumntype{R}[1]{>{\raggedleft\let\newline\\\arraybackslash\hspace{0pt}}m{#1}}

\definecolor{shadecolor}{RGB}{210,210,210}
\newcommand{\asm}[1]{\colorbox{shadecolor}{#1}}

\newcommand{\qrlinkframe}[2]{\begin{frame}{#1}
\center\qrcode[hyperlink,height=75px]{#2}
\end{frame}}

\title{Семинар 12: файлы в Linux}
\date{25 февраля, 2020}


\begin{document}

\begin{frame}
  \titlepage
\end{frame}

\begin{frame}[fragile]
\frametitle{Интерфейс Linux для файлов [x2]}
\begin{center}
    \begin{minipage}{0.95\textwidth}
        \begin{minted}{c}
#include <sys/types.h>
#include <unistd.h>
#include <fcntl.h>

off_t lseek(int fd, off_t offset, int whence);
        \end{minted}
    \end{minipage}
\end{center}
\end{frame}

\begin{frame}{lseek}
\begin{itemize}
    \item С каждым файловым дескриптором внутри ядра ассоциируется \mintinline{c}{struct file}
    \item У этой структуры есть поле, отвечающее за текущее положение в файле (\mintinline{c}{f_pos})
    \item За несколькими файловыми дескрипторами может быть одна \mintinline{c}{struct file}!
    \item lseek позволяет изменять позицию в файле
\end{itemize}
\end{frame}

\begin{frame}{lseek: whence}
\begin{itemize}
    \item \textbf{SEEK\_SET}: \mintinline{c}{f_pos = offset}
    \item \textbf{SEEK\_CUR}: \mintinline{c}{f_pos += offset}
    \item \textbf{SEEK\_END}: \mintinline{c}{f_pos = file_size + offset}
\end{itemize}
\end{frame}

\begin{frame}{lseek: off\_t}
\begin{itemize}
    \item Специальный тип, который предназначен для хранения оффсетов
    \item Может быть 32ух битным
    \item Чтобы было 64 бита: \mintinline{bash}{-D_FILE_OFFSET_BITS=64}
\end{itemize}
\end{frame}

\begin{frame}{Имена файлов}
\begin{itemize}
    \item Относительный путь отсчитывается от текущей директории
    \item Абсолютный путь — от корня
    \item В реальности длина ничем не ограничена
    \item Путь, передаваемый в сисколлы не может превосходить \mintinline{c}{PATH_MAX}
\end{itemize}
\end{frame}

\qrlinkframe{PATH\_MAX Is Tricky}{https://eklitzke.org/path-max-is-tricky}

\begin{frame}[fragile]
\frametitle{Мета-информация файлов}
\begin{center}
    \begin{minipage}{1\textwidth}
        \begin{minted}{c}
#include <sys/stat.h>
#include <unistd.h>

int stat(const char* pathname, struct stat* buf);
int lstat(const char* pathname, struct stat* buf);
int fstat(int fd, struct stat *buf);
ssize_t readlink(const char *pathname, char *buf, size_t bufsiz);
        \end{minted}
    \end{minipage}
\end{center}
\end{frame}

\begin{frame}[fragile]
\frametitle{struct stat}
    \begin{minipage}{1\textwidth}
        \begin{minted}{c}
#include <sys/stat.h>

struct stat {
   dev_t     st_dev;      /* ID of device containing file */
   ino_t     st_ino;      /* Inode number */
   mode_t    st_mode;     /* File type and mode */
   nlink_t   st_nlink;    /* Number of hard links */
   uid_t     st_uid;      /* User ID of owner */
   gid_t     st_gid;      /* Group ID of owner */
   dev_t     st_rdev;     /* Device ID (if special file) */
   off_t     st_size;     /* Total size, in bytes */
   blksize_t st_blksize;  /* Block size for filesystem I/O */
   blkcnt_t  st_blocks;   /* Number of allocated blocks */
   time_t st_atime;       /* Time of last access */
   time_t st_mtime;       /* Time of last modification */
   time_t st_ctime;       /* Time of last status change */
};
        \end{minted}
    \end{minipage}
\end{frame}

\begin{frame}{Типы файлов}
\begin{itemize}
    \item 7 типов файлов
    \item Регулярные (\textbf{S\_ISREG})
    \item Директории (\textbf{S\_ISDIR})
    \item Символьные устройства (\textbf{S\_ISCHR})
    \item Блочные устройства (\textbf{S\_ISBLK})
    \item Именованные пайпы (\textbf{S\_ISFIFO})
    \item Символические ссылки (\textbf{S\_ISLNK})
    \item Сокеты (\textbf{S\_ISSOCK})
\end{itemize}
\end{frame}

\begin{frame}{Права доступа}
\begin{itemize}
    \item 3 группы по 3 бита определяют права на чтение (r), запись (w) и исполнение (x)
    \item Чаще всего записываются в виде \mintinline{c}{rwxrwxrwx} или восьмеричных чисел
    \item Примеры:
    \item \mintinline{c}{777 = rwxrwxrwx} — права доступа кому угодно
    \item \mintinline{c}{666 = rw-rw-rw-} — права доступа на чтение запись кому угодно
    \item \mintinline{c}{700 = rw-------} — читать/писать может только владелец
    \item Сначала проверяются права на владельца, потом на группу и потом для остальных
    \item Если у владельца \mintinline{c}{---}, но у группы \mintinline{c}{rwx}, то у владельца доступа всё равно не будет, даже если он будет входить в группу-владельца
\end{itemize}
\end{frame}

\begin{frame}{inode и файловые ссылки}
\begin{itemize}
    \item inode — структура, которая содержит в себе мета-информацию (аттрибуты) и аллоцированные блоки
    \item Также под inode часто понимают её уникальный номер
    \onslide<2->\item Символические ссылки — файлы, которые содержат в себе пути до других файлов
    \onslide<2->\item Жёсткие ссылки — различные «имена» одного и того же файла
    \onslide<2->\item Жёсткие ссылки на один и тот же файл имеют одинаковые inode в отличие от символических
    \onslide<2->\item Жёсткие ссылки должны быть на одной файловой системе
    \onslide<2->\item Символические ссылки могут ссылаться на несуществующий файл
    \onslide<2->\item \mintinline{c}{lstat} не следует по символическим ссылкам в отличие от \mintinline{c}{stat}
\end{itemize}
\end{frame}

\begin{frame}[fragile]
\frametitle{Проверка прав доступа}
\begin{center}
    \begin{minipage}{0.95\textwidth}
        \begin{minted}{c}
#include <unistd.h>

#define F_OK ...
#define R_OK ...
#define W_OK ...
#define X_OK ...

int access(const char *pathname, int mode);
        \end{minted}
    \end{minipage}
\end{center}
\end{frame}


\begin{frame}[fragile]
\frametitle{File allocations}
\begin{center}
    \begin{minipage}{0.95\textwidth}
        \begin{minted}{c}
#include <sys/types.h>
#include <unistd.h>
#include <fcntl.h>

int truncate(const char* path, off_t length);
int ftruncate(int fd, off_t length);
int fallocate(int fd, int mode, off_t offset, off_t len);
        \end{minted}
    \end{minipage}
\end{center}
\end{frame}

\begin{frame}{ftruncate}
\begin{itemize}
    \item Позволяет изменять \emph{размер} файла, не используя \mintinline{c}{write}
    \item Если \mintinline{c}{file_size < length}, то добивает файл \mintinline{c}{'\0'}
    \item Если \mintinline{c}{file_size > length}, то данные теряются
    \item Не меняет \mintinline{c}{f_pos}
\end{itemize}
\end{frame}

\begin{frame}{fallocate}
\begin{itemize}
    \item Аллоцирует \emph{дисковое пространство}
    \item После этого вызова гарантируется, что для $[offset; offset + len)$ будет выделено реальное место на диске
\end{itemize}
\end{frame}

\begin{frame}[fragile]
\frametitle{Создание и удаление файлов}
\begin{center}
    \begin{minipage}{0.95\textwidth}
        \begin{minted}{c}
#include <sys/stat.h>

int link(const char *oldpath, const char *newpath);
int symlink(const char *oldpath, const char *newpath);
int mkdir(const char *pathname, mode_t mode);

int rmdir(const char *pathname);
int unlink(const char *oldpath, const char *newpath);
int rename(const char* oldpath, const char* newpath);
        \end{minted}
    \end{minipage}
\end{center}
\end{frame}

\begin{frame}[fragile]
\frametitle{Работа с директориями}
\begin{center}
    \begin{minipage}{0.95\textwidth}
        \begin{minted}{c}
#include <sys/stat.h>

DIR *opendir(const char *pathname);

struct dirent {
    ino_t d_ino;
    char d_name[NAME_MAX + 1];
}
struct dirent *readdir(DIR *dp);

void rewinddir(DIR *dp);
int closedir(DIR *dp);
long telldir(DIR *dp);
void seekdir(DIR *dp, long loc);
        \end{minted}
    \end{minipage}
\end{center}
\end{frame}

\begin{frame}[fragile]
\frametitle{Работа с рабочей директорией}
\begin{center}
    \begin{minipage}{0.95\textwidth}
        \begin{minted}{c}
#include <unistd.h>

char *getcwd(char *buf, size_t size);
int chdir(const char *pathname);
int fchdir(int filedes);
        \end{minted}
    \end{minipage}
\end{center}
\end{frame}


\qrlinkframe{Работа со временем}{https://github.com/hseos/hseos-course/tree/master/2018/13-files2\#\%D1\%80\%D0\%B0\%D0\%B1\%D0\%BE\%D1\%82\%D0\%B0-\%D1\%81\%D0\%BE-\%D0\%B2\%D1\%80\%D0\%B5\%D0\%BC\%D0\%B5\%D0\%BD\%D0\%B5\%D0\%BC}


\begin{frame}
\center\Huge{Gratias ago!}
\end{frame}


\end{document}
