\documentclass[10pt,pdf,hyperref={unicode}]{beamer}

\usepackage[normalem]{ulem}
\usepackage{qrcode}
\usepackage{array}
\usepackage[T2A]{fontenc}
\usepackage[utf8]{inputenc}
\usepackage{colortbl}
\usepackage{minted}
\usepackage{listings}
\usepackage{tcolorbox}

\setbeamertemplate{navigation symbols}{}

\usetheme{default}

\usepackage{array}
\newcolumntype{L}[1]{>{\raggedright\let\newline\\\arraybackslash\hspace{0pt}}m{#1}}
\newcolumntype{C}[1]{>{\centering\let\newline\\\arraybackslash\hspace{0pt}}m{#1}}
\newcolumntype{R}[1]{>{\raggedleft\let\newline\\\arraybackslash\hspace{0pt}}m{#1}}

\definecolor{shadecolor}{RGB}{210,210,210}
\newcommand{\asm}[1]{\colorbox{shadecolor}{#1}}

\newcommand{\qrlinkframe}[2]{\begin{frame}{#1}
\center\qrcode[hyperlink,height=75px]{#2}
\end{frame}}

\title{Семинар 13: немного о линкерах}
\date{10 марта, 2020}


\begin{document}

\begin{frame}
  \titlepage
\end{frame}

\begin{frame}{Основные понятия}
\begin{itemize}
    \item \textbf{Symbol} — именованная сущность (обычно функция или глобальная переменная), за которой должен быть закреплён адрес
    \item \textbf{Content} — описание того, как ОС должна загрузить символ в память
    \item \textbf{Relocation} — указание вычисления адреса по определённому правилу
    \item \textbf{Object file} — файл-контейнер для символов и их содержимого
    \item Релокации обычно состоят из \textbf{offset} и \textbf{addend}
\end{itemize}
\end{frame}

\begin{frame}{Что происходит во время статической линковки?}
\begin{enumerate}
    \item Чтение и парсинг объектных файлов
    \item Построение таблицы символов
    \item Размещение всех известных секций в памяти
    \item Применение релокаций
\end{enumerate}
\end{frame}

\qrlinkframe{Спецификация ELF}{http://www.sco.com/developers/devspecs/gabi41.pdf}
\qrlinkframe{Спецификация ELF для x86-64}{http://refspecs.linuxbase.org/elf/x86_64-abi-0.99.pdf}

\begin{frame}[fragile]
\frametitle{Описание релокации в ELF}
\begin{center}
    \begin{minipage}{0.95\textwidth}
        \begin{minted}{c}
typedef struct {
    Elf32_Addr r_offset;
    Elf32_Addr r_info;
    Elf32_Addr r_addend;
} Elf32_Rela;
        \end{minted}
    \end{minipage}
\end{center}
\end{frame}

\begin{frame}{Релокации в x86-64}
\begin{itemize}
    \item На данный момент для x86-64 существует \textbf{38} возможных релокаций
    \item Самые распространённые: \textbf{R\_X86\_64\_PC32}, \textbf{R\_X86\_64\_GOT32}, \textbf{R\_X86\_64\_PLT32}
    \item Полный список можно найти в ABI для x86-64
\end{itemize}
\end{frame}

\begin{frame}{Динамическая линковка}
\begin{itemize}
    \item Исторически идея была придумана для экономии памяти
    \item Давайте вынесем часто используемые функции в разделяемые библиотеки или \textbf{shared objects}
    \item Эти объекты будем подклеивать в различные виртуальные адресные пространства, но физически их будем хранить в единственном экземпляре
    \item Как устроить процесс линковки для таких библиотек?
\end{itemize}
\end{frame}

\begin{frame}{Динамическая линковка}
\begin{itemize}
    \item Position independent code
    \item Global offset table
    \item Procedure linkage table
\end{itemize}
\end{frame}

\begin{frame}{Position independent code}
\begin{itemize}
    \item Код, начало которого может располагаться как угодно в памяти
    \item Значит, что нельзя использовать абсолютные адреса, только относительные
    \item RIP-relative addressing
    \item Кроме разделяемых объектов используется для \emph{address space layout randomization} (ASLR)
\end{itemize}
\end{frame}

\begin{frame}{Global offset table}
\begin{itemize}
    \item Специальная секция (\mintinline{c}{.got} и \mintinline{c}{.got.plt}) в ELF
    \item Хранит в себе отображение из названий символов в адреса
    \item При динамической линковке эти адреса заполняются специальным \emph{интерпретатором} (\mintinline{c}{ld-linux.so})
    \item Релокации применяются на стадии линковки, но ссылаются на секцию, которая будет заполнена во время исполнения
    \item Однако, это всё равно называются релокациями, но они уже находятся в других секциях (\mintinline{c}{.rel.dyn}, \mintinline{c}{.rel.plt})
\end{itemize}
\end{frame}

\begin{frame}{Procedure linkage table}
\begin{itemize}
    \item Секция, которая содержит специальным образом сформированные заглушки для вызова процедур
    \item Все вызовы сторонних функций в реальности делают \mintinline{gas}{call} на символы, объявленные в этой секции
    \item Первый вызов будет триггерить динамический линковщик, остальные вызовы будут напрямую вызывать функцию (\emph{lazy binding})
    \item Можно отключить с помощью опции линкера: \mintinline{c}{-z now}
\end{itemize}
\end{frame}

\begin{frame}[fragile]
\frametitle{Устройство заглушки в PLT-секции: до резолва}
\begin{center}
    \begin{minipage}{0.95\textwidth}
        \begin{minted}[mathescape=true]{gas}
FUNC@plt:
    jmpq   *FUNC_GOT_OFFSET(%rip)  # Прыжок по адресу в GOT
    pushq  index  # Индекс символа в .rel.dyn
    jmpq   FIRST_PLT_ENTRY  # relative jump

FUNC@got.plt: # адрес pushq

        \end{minted}
    \end{minipage}
\end{center}
\end{frame}

\begin{frame}{Процесс резолва}
\begin{enumerate}
    \item Вызов \mintinline{text}{FUNC@plt}
    \item Прыжок по адресу в GOT
    \item Возврат обратно в PLT
    \item Подготовка аргументов для линкера
    \item Переход в FIRST\_PLT\_ENTRY
    \item Вызов линкера
    \item Поиск .so-файла и его подгрузка через mmap + mprotect
    \item Запись адреса FUNC в FUNC@got.plt
    \item Вызов FUNC
\end{enumerate}
\end{frame}

\begin{frame}[fragile]
\frametitle{Устройство заглушки в PLT-секции: после резолва}
\begin{center}
    \begin{minipage}{0.95\textwidth}
        \begin{minted}[mathescape=true]{gas}
FUNC@plt:
    jmpq   *FUNC_GOT_OFFSET(%rip)  # Прыжок по адресу в GOT
    # инструкции ниже никогда больше не выполнятся
    pushq  index
    jmpq   FIRST_PLT_ENTRY

FUNC@got.plt: # абсолютный адрес FUNC
        \end{minted}
    \end{minipage}
\end{center}
\end{frame}

\begin{frame}{Оставшиеся вопросы}
\begin{itemize}
    \item Как линкер узнает, какие shared objects нужно подгрузить? \onslide<2->{Из секции dynamic, которую он парсит на старте}
    \item Как получить текущий EIP под x86?
    \item Зачем разделять GOT и PLT?
\end{itemize}
\end{frame}

\begin{frame}
\center\Huge{Как получить текущий EIP под x86?}
\end{frame}

\begin{frame}[fragile]
\frametitle{\_\_x86.get\_pc\_thunk.bx}
\begin{center}
    \begin{minipage}{0.95\textwidth}
        \begin{minted}[mathescape=true]{gas}
__x86.get_pc_thunk.bx:
    mov (%esp), %ebx
    ret

    ...

    call __x86.get_pc_thunk.bx
    # теперь ebx содержит текущий EIP
        \end{minted}
    \end{minipage}
\end{center}
\end{frame}

\begin{frame}{Зачем разделять GOT и PLT?}
\begin{itemize}
    \item Indirect call — 6 байт, direct call — 5 байт
    \item Поэтому компилятор должен был бы различать для чего он компилирует код: для динамической линковки или статической
    \item На x86-64 Можно использовать RIP-relative addressing и \mintinline{c}{-fno-plt}
\end{itemize}
\end{frame}

\begin{frame}{Статическая vs динамическая линковка}
\begin{itemize}
    \item Динамическая линковка вызывает \emph{dependency hell}
    \item Простота эксплуатации > экономия памяти
    \item Молодые языки (например, Rust и Go), поэтому используют статическую линковку
    \item Но из-за этого растёт размер исполняемых файлов и время компиляции
\end{itemize}
\end{frame}

\qrlinkframe{Ian Lance Taylor about linkers}{https://www.airs.com/blog/archives/38}

\begin{frame}{vDSO}
\begin{itemize}
    \item Virtual dynamic shared object
    \item Полноценный ELF-объект внутри каждого процесса
    \item Ускоряет вызов сисколлов, которые не требуют доступа к <<тяжёлым>> данным ядра
    \item На данный момент содержит в себе: \mintinline{c}{gettimeofday}, \mintinline{c}{clock_gettime}, \mintinline{c}{getcpu}.
    \item Идея состоит в том, чтобы реализацию сисколла унести в userspace, а необходимые данные периодически подкладывать в адресное пространство процесса
\end{itemize}
\end{frame}

\begin{frame}
\center\Huge{Gratias ago!}
\end{frame}


\end{document}
