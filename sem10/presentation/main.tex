\documentclass[10pt,pdf,hyperref={unicode}]{beamer}

\usepackage[normalem]{ulem}
\usepackage{qrcode}
\usepackage{array}
\usepackage[T2A]{fontenc}
\usepackage[utf8]{inputenc}
\usepackage{colortbl}
\usepackage{minted}
\usepackage{listings}
\usepackage{tcolorbox}

\setbeamertemplate{navigation symbols}{}

\usetheme{default}

\usepackage{array}
\newcolumntype{L}[1]{>{\raggedright\let\newline\\\arraybackslash\hspace{0pt}}m{#1}}
\newcolumntype{C}[1]{>{\centering\let\newline\\\arraybackslash\hspace{0pt}}m{#1}}
\newcolumntype{R}[1]{>{\raggedleft\let\newline\\\arraybackslash\hspace{0pt}}m{#1}}

\definecolor{shadecolor}{RGB}{210,210,210}
\newcommand{\asm}[1]{\colorbox{shadecolor}{#1}}

\newcommand{\qrlinkframe}[2]{\begin{frame}{#1}
\center\qrcode[hyperlink,height=75px]{#2}
\end{frame}}

\title{Семинар 10: Intel x86 assembly. Часть 3}
\date{27 января, 2020}


\begin{document}

\begin{frame}
  \titlepage
\end{frame}

\begin{frame}{В прошлых сериях...}
\begin{itemize}
    \item Registers
    \item Instructions
    \item Memory dereferencing
    \item Stack
    \item Calling conventions: x86 \& x86-64
\end{itemize}
\end{frame}

\begin{frame}[fragile]
\frametitle{План на сегодня}
\begin{center}
    \begin{minipage}{0.65\textwidth}
        \begin{minted}{c}
float hypot_x86(float x, float y);
float hypot_x86_64(float x, float y);
        \end{minted}
    \end{minipage}
\end{center}
\end{frame}

\begin{frame}{x87: немного истории}
\begin{itemize}
    \item Intel 80387 (1987 год) — первый сопроцессор Intel, поддерживающий IEEE754
    \item Имелся отдельный набор команд (instruction set) для процессоров x86
    \item С 1991 года расположен на кристалле процессора
\end{itemize}
\end{frame}

\begin{frame}{x87: технические детали}
\begin{itemize}
    \item x87 имел \textbf{восемь} 64-битных регистров: \asm{st(0)}-\asm{st(1)}
    \item Регистры образуют стек, \asm{st(0)} -- <<вершина>>
    \item Инструкции могли оперировать с любыми регистрами
    \item Но x87 позволял загружать числа из памяти на только в вершину стека и извлекать только из неё в память
\end{itemize}
\end{frame}


\begin{frame}[fragile]
\frametitle{x87: пример}
\begin{center}
    \begin{minipage}{0.65\textwidth}
        \begin{minted}[mathescape=true]{gas}
A: .float 4195835
B: .float 3145727
    # ...
    fld B  # st(0) = B
    fld A  # st(0) = A, st(1) = B
    fdivp  # st(0) $\approx$ 1.33382
        \end{minted}
    \end{minipage}
\end{center}
\end{frame}


\qrlinkframe{Pentium FDIV bug}{https://en.wikipedia.org/wiki/Pentium_FDIV_bug}

\begin{frame}
\center\Huge{Зачем нужно знать про x87?}
\end{frame}

\begin{frame}{Соглашения о вызовах под x86}
\begin{itemize}
    \item \textbf{float} и \textbf{double} аргументы передаются \textbf{по стеку}
    \item Возвращаемое значение (\textbf{float} или \textbf{double}) лежит в \asm{st(0)}
    \item \asm{st(1)}-\asm{st(7)} \textbf{должны быть пустыми} в момент входа функцию и в момент выхода из неё
\end{itemize}
\end{frame}

\begin{frame}{SSE}
\begin{itemize}
    \item SSE -- \emph{Streaming SIMD Extensions}
    \item SIMD -- \emph{Single Instruction Multiple Data}
    \item Набор команд для процессоров Intel
    \item Впервые появился в Pentium III (1999 год)
    \item Расширения: SSE2, SSE3, SSE4
\end{itemize}
\end{frame}

\begin{frame}{SSE}
\begin{itemize}
    \item Добавляет в процессор 8 регистров: \asm{xmm0}-\asm{xmm7}
    \item Размер каждого -- 128 битов
    \item Два типа данных -- float (single) и double
    \item Два режима -- scalar и packed
\end{itemize}
\end{frame}

\begin{frame}{SSE: скалярные инструкции}
\begin{itemize}
    \item Последние две буквы отвечают за тип инструкции
    \item xxxSS = scalar single, то есть операция над одним float'ом
    \item xxxSD = scalar double, то есть операция над одним double'ом
    \item Например: \asm{addss}, \asm{subsd}, \asm{divss}
\end{itemize}
\end{frame}

\begin{frame}{SSE: packed инструкции}
\begin{itemize}
    \item xxxPS = packed single
    \item xxxPD = packed double
    \item Оперируют сразу несколькими числами в одном регистре: 4 float или 2 double
    \item Отсюда и название -- \emph{single instruction multiple data}
\end{itemize}
\end{frame}

\begin{frame}{SSE: чтение данных}
\begin{itemize}
    \item \asm{movupX} читает невыровненные данные
    \item \asm{movapX} -- выровненные по 16 байт
    \item Для скалярных типов используются \asm{movsX}, которые не требуют выравнивания
\end{itemize}
\end{frame}

\begin{frame}{Соглашения о вызовах под x86-64}
\begin{itemize}
    \item \textbf{float} и \textbf{double} аргументы передаются в \asm{xmm0}-\asm{xmm7}
    \item Если больше 8 аргументов, то остальные -- по стеку
    \item Возвращаемое значение (\textbf{float} или \textbf{double}) лежит в \asm{xmm0}
    \item Все xmm-регистры являются caller-saved
\end{itemize}
\end{frame}

\begin{frame}[fragile]
\frametitle{Если очень хочется использовать SSE под x86}
\begin{center}
    \begin{minipage}{0.9\textwidth}
        \begin{minted}[mathescape=true]{gas}
    sub $8, %esp         # немного места на стеке
    movsd %xmm0, (%esp)  # сохраняем результат на стек
    fldl (%esp)          # сохраняем рехультат в st(0)
    add $8, %esp
        \end{minted}
    \end{minipage}
\end{center}
\end{frame}

\begin{frame}{Intrinsics}
    \begin{itemize}
        \item Иногда писать на голом ассемблере очень трудно
        \item Intel разработал специальные расширения для компиляторов — intrinsics
        \item В каждом компиляторе реализованы по-разному
        \item Можно писать код на С, оптимизируя медленные места с помощью C-подобных функций
    \end{itemize}
\end{frame}

\qrlinkframe{Intel Intrinsics Guide}{https://software.intel.com/sites/landingpage/IntrinsicsGuide/}

\begin{frame}[fragile]
\frametitle{Как сравнивать вещественные числа?}
С помощью comisX для SSE:
\begin{center}
    \begin{minipage}{0.9\textwidth}
        \begin{minted}[mathescape=true]{gas}
    comiss %xmm0, %xmm1  # сравнивает %xmm0 с %xmm1
    ja .greater
    # ...
        \end{minted}
    \end{minipage}
\end{center}

И fcomi для x87:

\begin{center}
    \begin{minipage}{0.9\textwidth}
        \begin{minted}[mathescape=true]{gas}
    fcomi
    ja .greater  # сравнивает st(0) с st(1)
    # ...
        \end{minted}
    \end{minipage}
\end{center}
\end{frame}

\begin{frame}
\center\Huge{Дякую!}
\end{frame}


\end{document}
