\documentclass[10pt,pdf,hyperref={unicode}]{beamer}

\usepackage[normalem]{ulem}
\usepackage{qrcode}
\usepackage{array}
\usepackage[T2A]{fontenc}
\usepackage[utf8]{inputenc}
\usepackage{colortbl}
\usepackage{minted}
\usepackage{listings}
\usepackage{tcolorbox}

\setbeamertemplate{navigation symbols}{}

\usetheme{default}

\usepackage{array}
\newcolumntype{L}[1]{>{\raggedright\let\newline\\\arraybackslash\hspace{0pt}}m{#1}}
\newcolumntype{C}[1]{>{\centering\let\newline\\\arraybackslash\hspace{0pt}}m{#1}}
\newcolumntype{R}[1]{>{\raggedleft\let\newline\\\arraybackslash\hspace{0pt}}m{#1}}

\definecolor{shadecolor}{RGB}{210,210,210}
\newcommand{\asm}[1]{\colorbox{shadecolor}{#1}}

\newcommand{\qrlinkframe}[2]{\begin{frame}{#1}
\center\qrcode[hyperlink,height=75px]{#2}
\end{frame}}

\title{Семинар 11: Intel x86 assembly. Часть 4}
\date{11 февраля, 2020}


\begin{document}

\begin{frame}
  \titlepage
\end{frame}

\qrlinkframe{Материалы семинаров на GitHub}{https://github.com/carzil/hse-caos-2019-seminars}

\begin{frame}{Системные вызовы}
\begin{itemize}
    \item Способ общения с операционной системой
    \item В C выглядят как обычные функции
    \item Реализуются libc
    \item В ассемблере всё хитрее
\end{itemize}
\end{frame}

\begin{frame}{Системные вызовы: механизм работы}
\begin{itemize}
    \item Все программы работают в непривилегированном режиме (ring3)
    \item В этом режиме нельзя напрямую работать с физической памятью, устройствами и портами ввода-вывода
    \item Как перейти в привилегированный режим (ring0)?
\end{itemize}
\end{frame}

\begin{frame}{Прерывания процессора}
\begin{itemize}
    \item Прерывания — специальные события, которые «прерывают» исполнение команд
    \item Прерывание может произойти после любой инструкции, но не во время её выполнения
    \item Interrupt vector table — массив указателей на функции-обработчики прерываний (x86)
\end{itemize}
\end{frame}

\begin{frame}{Прерывания процессора}
\begin{itemize}
    \item Hardware interrupts — железные события
    \item Software interrupts — события, сгенерированные программами
    \item Exceptions — исключительные ситуации процессора
\end{itemize}
\end{frame}

\begin{frame}{Примеры прерываний}
\begin{itemize}
    \item События сетевой карты, клавиатуры, видеокарты, ...
    \item \asm{int} и \asm{syscall}
    \item Divide-by-Zero
    \item Page fault (double fault, triple fault)
    \item General Protection Fault
    \item Invalid Opcode
    \item x87 Floating-Point Exception
    \item SIMD Floating-Point Exception
\end{itemize}
\end{frame}

\begin{frame}{Системные вызовы: x86}
\begin{itemize}
    \item Порядковый номер вызова записывается в \textbf{eax}
    \item Аргументы передаются в \textbf{регистрах}: \textbf{ebx}, \textbf{ecx}, \textbf{edx}, \textbf{esi}, \textbf{edi}
    \item Затем делается \asm{int 0x80}
\end{itemize}
\end{frame}

\qrlinkframe{Таблица системных вызовов Linux x86}{https://syscalls.kernelgrok.com/}

\begin{frame}{Системные вызовы: x86-64}
\begin{itemize}
    \item Порядковый номер вызова записывается в \textbf{rax}
    \item Аргументы передаются в \textbf{регистрах}: \textbf{rdi}, \textbf{rsi}, \textbf{rdx}, \textbf{r10}, \textbf{r8}, \textbf{r9}
    \item Затем делается \asm{syscall}
\end{itemize}
\end{frame}

\qrlinkframe{Таблица системных вызовов Linux x86-64}{https://filippo.io/linux-syscall-table/}

\begin{frame}{Red zone}
\begin{itemize}
    \item Область размером 128 байт \textbf{под} текущим \textbf{rsp}
    \item System V AMD64 ABI гарантирует, что это место не используется обработчиком прерывания
    \item Используется для оптимизации, например, последняя функция может не выделять полноценный фрейм, а использовать red zone
\end{itemize}
\end{frame}

\qrlinkframe{Что будет, если забыть про red zone?}{https://alex.dzyoba.com/blog/redzone/}

\begin{frame}{Файлы}
\begin{itemize}
    \item Каждый файл имеет путь относительно /
    \item Передавать каждый раз путь в системный вызов -- затратно
    \item Поэтому (и не только) изобрели \emph{файловые дескрипторы}
    \item Файловый дескриптор -- какое-то число, которое идентифицирует файл
\end{itemize}
\end{frame}

\begin{frame}
\center\huge{Правило \#1: всегда закрывать файловые дескрипторы, после завершения работы с ними!}
\end{frame}

\begin{frame}{Файловые дескрипторы}
\begin{itemize}
    \item 0 -- stdin, 1 -- stdout, 2 -- stderr
    \item Выдаются последовательно
    \item Под файловым дескриптором может быть что угодно: регулярные файлы, именованные каналы, сокеты, linux-specific интерфейсы (epoll, signalfd, eventfd), ...
    \item Ограниченное количество, лимит задаётся на каждый процесс, увеличить можно с помощью ulimit
    \item Утекание файловых дескрипторов намного страшнее утечек памяти
\end{itemize}
\end{frame}

\begin{frame}[fragile]
\frametitle{Интерфейс Linux для файлов}
\begin{center}
    \begin{minipage}{0.95\textwidth}
        \begin{minted}{c}
#include <unistd.h>
#include <fcntl.h>
int open(const char *pathname, int oflag, mode_t mode);
ssize_t read(int fd, void *buf, size_t nbytes);
ssize_t write(int fd, const void *buf, size_t nbytes);
int close(int fd);
        \end{minted}
    \end{minipage}
\end{center}
\end{frame}

\begin{frame}{open}
\begin{itemize}
    \item «Открывает» файл с заданным режимом (на чтение/запись) и возвращает файловый дескриптор
    \item \textbf{O\_RDONLY} -- только чтение
    \item \textbf{O\_WRONLY} -- только запись
    \item \textbf{O\_RDWR} -- и чтение, и запись
    \item \textbf{O\_CREAT} -- создать файл, если он не существует (mode в этом случае -- права на файл)
    \item \textbf{O\_EXCL} -- используется вместе \textbf{O\_CREAT} и тогда гарантирует атомарность создания файла
\end{itemize}
\end{frame}

\begin{frame}{read}
\begin{itemize}
    \item Копирует \textbf{ИЗ} файлового дескриптора \textbf{fd} \emph{не более} \textbf{nbytes} в буфер \textbf{buf}
    \item Возвращает 0, если больше нечего <<читать>>
    \item Если произошла ошибка возвращает -1 и выставляет \textbf{errno}
    \item В противном случае заблокируется, пока данные не будут доступны на чтение
    \item Возвратит количество прочитанных байт
\end{itemize}
\end{frame}

\begin{frame}{write}
\begin{itemize}
    \item Копирует \textbf{в} файловый дескриптор \textbf{fd} \emph{не более} \textbf{nbytes} из буфера \textbf{buf}
    \item Возвращает -1, если произошла ошибка
    \item Возвращает количество записанных байт (всегда >0)
    \item Может заблокироваться, если файл под файловым дескриптором пока нельзя что-либо записать
\end{itemize}
\end{frame}

\begin{frame}{close}
\begin{itemize}
    \item Закрывает (освобождает) файловый дескриптор
    \item Возвращает -1, если произошла ошибка -- это бывает очень редко
\end{itemize}
\end{frame}

\begin{frame}{Inline assembly}
\begin{itemize}
    \item Позволяют встраивать код на ассемблере внутрь С/C++
    \item Используют подстановки в специальном синтаксисе­
    \item Обычно используются только в крайнем случае, когда нужно вставить особую процессорную инструкцию в тело функции (например, инструкцию, которая запрещает прерывания)
\end{itemize}
\end{frame}

\begin{frame}[fragile]
\frametitle{Как это \sout{отвратительно} выглядит}
\begin{center}
    \begin{minipage}{0.9\textwidth}
        \begin{minted}{c}
asm ("mov %3, %0\n"
     "mov %2, %1\n"
     : "=r" (a), "=m"(b)     // выходные параметры
     : "r" (c), "m"(d)       // входные параметры
     : "rax", "rbx", "rdx")  // clobbers
        \end{minted}
    \end{minipage}
\end{center}
\end{frame}

\qrlinkframe{Мануал по inline assembly}{https://gcc.gnu.org/onlinedocs/gcc/Extended-Asm.html}

\begin{frame}
\center\Huge{Дзякуй!}
\end{frame}


\end{document}
