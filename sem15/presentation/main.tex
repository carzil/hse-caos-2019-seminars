\documentclass[10pt,pdf,hyperref={unicode}]{beamer}

\usepackage[normalem]{ulem}
\usepackage{qrcode}
\usepackage{array}
\usepackage[T2A]{fontenc}
\usepackage[utf8]{inputenc}
\usepackage{colortbl}
\usepackage{minted}
\usepackage{listings}
\usepackage{tcolorbox}

\setbeamertemplate{navigation symbols}{}

\usetheme{default}

\usepackage{array}
\newcolumntype{L}[1]{>{\raggedright\let\newline\\\arraybackslash\hspace{0pt}}m{#1}}
\newcolumntype{C}[1]{>{\centering\let\newline\\\arraybackslash\hspace{0pt}}m{#1}}
\newcolumntype{R}[1]{>{\raggedleft\let\newline\\\arraybackslash\hspace{0pt}}m{#1}}

\definecolor{shadecolor}{RGB}{210,210,210}
\newcommand{\asm}[1]{\colorbox{shadecolor}{#1}}

\newcommand{\qrlinkframe}[2]{\begin{frame}{#1}
\center\qrcode[hyperlink,height=75px]{#2}
\end{frame}}

\title{Семинар 15: процессы}
\date{26 марта, 2020}


\begin{document}

\begin{frame}
  \titlepage
\end{frame}

\begin{frame}{Процессы}
\begin{itemize}
    \item Сам термин «процесс» довольно сложно определить
    \item POSIX: <<\emph{A process is an abstraction that represents an executing program. Multiple processes execute independently and have separate address spaces. Processes can create, interrupt, and terminate other processes, subject to security restrictions.}>>
    \item В самом ядре Linux нет понятия «процесс», вместо этого оно оперирует «тасками»
    \item То, что в POSIX процесс, в Linux называется thread group
    \item Дальше мы всё таки будем говорить процессы :)
    \item Процессы объединяются в \emph{группы процессов}
    \item Группы процессов объединяются в \emph{сессии}
\end{itemize}
\end{frame}

\begin{frame}{Аттрибуты процесса}
\begin{itemize}
    \item Сохранённых контекст процесса (регистры)
    \item Отображения памяти (VMA) и стек для ядра
    \item Файловые дескрипторы
    \item Current working directory (cwd) и текущий корень (\mintinline{c}{man 2 chroot})
    \item umask
    \item PID, PPID, TID, TGID, PGID, SID
    \item Resource limits
    \item Priority
    \item Capabilities
    \item Namespaces
\end{itemize}
\end{frame}

\begin{frame}{Аттрибуты процесса: PID, PPID, TGID, ...}
\begin{itemize}
    \item PID = process ID
    \item PPID = parent process ID
    \item TID = thread ID
    \item TGID = thread group ID
    \item PGID = process group ID
    \item SID = session ID
    \item \mintinline{c}{pid_t getpid()}, \mintinline{c}{pid_t getppid()}, \mintinline{c}{pid_t gettid()}
    \item \mintinline{c}{pid_t getpgid()}, \mintinline{c}{int setpgid(pid_t pid, pid_t pgid)}
    \item \mintinline{c}{pid_t setsid()}, \mintinline{c}{pid_t getsid(pid_t pid)}
    \item Всё это можно найти в \mintinline{c}{/proc/<pid>/status} или в \mintinline{c}{/proc/<pid>/stat}
\end{itemize}
\end{frame}

\begin{frame}{Аттрибуты владельца процесса}
\begin{itemize}
    \item UID (user ID или real user ID) — ID владельца процесса, \mintinline{c}{void setuid(uid_t)}
    \item EUID (effective user ID) используется для проверок доступа, \mintinline{c}{void seteuid(uid_t)}
    \item SUID (saved user ID) используется, чтобы можно было временно понизить привилегии
    \item FSUID (file system user ID) обычно совпадает с EUID, но может быть отдельно изменён через \mintinline{c}{int setfsuid(uid_t fsuid)}
    \item Непривилегированный процесс может выставлять EUID равный только в SUID, UID или опять в EUID
    \item Также есть понятие setuid/setgid флагов (правильно они называются sticky-флагами), в отличие от обычных файлов, владельцем такого процесса станет \emph{владелец файла}, а не текущий пользователь
    \item Есть аналогичные GID, EGID, SGID, FSGID
\end{itemize}
\end{frame}

\begin{frame}{Приоритет и nice процессов}
\begin{itemize}
    \item В Linux есть понятие приоритета процесса
    \item Всего существует 140 различных приоритетов двух типов: real-time priroity и nice
    \item Приоритет реального времени от 1 до 99
    \item Приоритет для пользователей (через nice): от 100 до 139
    \item $priority = 20 + niceValue$
    \item Чем выше nice value, тем менее он приоритетен (более вежливый) и тем реже он выполняется
    \item По-умолчанию nice value равен 0
\end{itemize}
\end{frame}

\begin{frame}{Лимиты ресурсов процессов}
\begin{itemize}
    \item Лимиты деляется на два типа: soft и hard
    \item Soft-лимит или текущий лимит -- по нему вычисляются проверки
    \item Hard-лимит — максимальное значение soft-лимита
    \item CPU-время, если процесс его превысит ему ядро отошлёт \textbf{SIGXCPU}, если превысит hard limit, то \textbf{SIGKILL}
    \item Размер записываемых файлов: write будет возвращать \textbf{EFBIG}
    \item Размер записываемых coredump-файлов
    \item На максимальный размер виртуальной памяти, выделенной процессу
    \item Количество одновременных процессов пользователя
    \item Количество файловых дескрипторов
    \item И ещё кучу всего :)
\end{itemize}
\end{frame}

\begin{frame}[fragile]
\frametitle{Лимиты ресурсов процессов}
\begin{center}
\begin{minipage}{0.95\textwidth}
\begin{minted}{c}
#include <sys/time.h>
#include <sys/resource.h>

int getrlimit(int resource, struct rlimit *rlim);
int setrlimit(int resource, const struct rlimit *rlim);

int prlimit(pid_t pid, int resource,
     const struct rlimit *new_limit,
           struct rlimit *old_limit);
\end{minted}
\end{minipage}
\end{center}
\end{frame}

\begin{frame}{Механизмы изоляции}
\begin{itemize}
    \item \mintinline{c}{man 7 namespaces} и \mintinline{c}{man 7 cgroups}
    \item Linux namespaces изолируют части отдельные процессов
    \item Существующие неймспейсы: user, network, mount, cgroup, pid и ut
    \item Механизм Linux для ограничения ресурсов процессов: в основном это более тонкие ограничения для потребления процессорного времени и памяти
\end{itemize}
\end{frame}

\begin{frame}{Работа с процессами: fork}
\begin{itemize}
    \item В Linux выбран не самый обычный подход для запуска новых процессов
    \item Создать новый процесс можно только \emph{скопировав} текущий с помощью \mintinline{c}{pid_t fork()}
    \item fork выйдет в двух процессах одновременно, но в однём вернёт PID ребёнка, а в другом — 0.
    \item Ребёнок будет полностью идентичен родителю, но файловые дескрипторы и адресное пространство будут \emph{скопированы}
    \item Копирование адресного пространства довольно затратная штука, поэтому используется copy-on-write
    \item Реально копироваться страница будет только при первой записи в родителе или ребёнке
\end{itemize}
\end{frame}

\qrlinkframe{Dirty COW (CVE-2016-5195)}{https://chao-tic.github.io/blog/2017/05/24/dirty-cow}

\begin{frame}[fragile]
\frametitle{Работа с процессами: fork}
\begin{center}
\begin{minipage}{0.95\textwidth}
\begin{minted}{c}
#include <unistd.h>

pid_t pid = fork();
if (pid == -1) {
    // fork сломался
} else if (pid == 0) {
    // ребёнок
    // текущий pid можно получить через getpid
} else {
    // родитель, pid содержит pid ребёнка
}
\end{minted}
\end{minipage}
\end{center}
\end{frame}

\begin{frame}{Работа с процессами: execve}
\begin{itemize}
    \item Создавать копии недостаточно, нужно уметь запускать произвольные файлы
    \item Для этого используется системный вызовов execve
    \item Он заменяет текущий процесс, процессом, созданным из указанного файла
    \item Это называется заменой образа процесса: заменяются только части адресного пространства
    \item Аттрибуты процесса сохраняются (в том числе: лимиты, неймспейсы, всевозможные *ID, ...), хотя бывают исключения
    \item Также в новом образе процесса останутся текущие файловые дескрипторы, не помеченные флагом \mintinline{c}{O_CLOEXEC}
\end{itemize}
\end{frame}

\begin{frame}[fragile]
\frametitle{Работа с процессами: execve}
\begin{center}
\begin{minipage}{0.95\textwidth}
\begin{minted}{c}
#include <unistd.h>

extern char **environ;
int execl(const char *path, const char *arg0,
        ... /*, (char *)0 */);
int execle(const char *path, const char *arg0,
        ... /*, (char *)0, char *const envp[]*/);
int execlp(const char *file, const char *arg0,
        ... /*, (char *)0 */);
int execv(const char *path, char *const argv[]);
int execve(const char *path, char *const argv[],
        char *const envp[]);
int execvp(const char *file, char *const argv[]);
int fexecve(int fd, char *const argv[], char *const envp[]);
\end{minted}
\end{minipage}
\end{center}
\end{frame}

\begin{frame}{Работа с процессами: exit}
\begin{itemize}
    \item Служит для завершения текущего процесса
    \item \mintinline{c}{exit} vs \mintinline{c}{_exit}
    \item Чтобы завершить текущую thread group можно воспользоваться \mintinline{c}{exit_group}
    \item После вызова происходит закрытие всех открытых файловых дескрипторов, освобождение страниц в физической памяти
    \item Если у процесса были дети, то их родителем станет процесс с PID == 1.
\end{itemize}
\end{frame}

\begin{frame}{Работа с процессами: wait}
\begin{itemize}
    \item Кроме запуска процессов нужно ещё и дожидаться их завершения
    \item Для этого используются системные вызовы семейства wait*
    \item Они дожидаются завершения процесса (конкретного или любого) и возвращают специальный \emph{exit status}
    \item Обычно exit status содержит то, что передали в \mintinline{c}{exit}, но иногда процесс может завершиться не сам, а с помощью сигнала
    \item Для того, чтобы различать такие случаи, используются специальные макросы
    \item Зомби-процесс (Z) — такой процесс, который завершился, но wait в родителе ещё не собрал информацию из него
\end{itemize}
\end{frame}


\begin{frame}[fragile]
\frametitle{Работа с процессами: wait}
\begin{center}
\begin{minipage}{0.95\textwidth}
\begin{minted}{c}
#include <sys/types.h>
#include <sys/time.h>
#include <sys/resource.h>
#include <sys/wait.h>

pid_t wait(int *wstatus);
pid_t waitpid(pid_t pid, int *wstatus, int options);

pid_t wait3(int *wstatus, int options,
           struct rusage *rusage);

pid_t wait4(pid_t pid, int *wstatus, int options,
           struct rusage *rusage);
\end{minted}
\end{minipage}
\end{center}
\end{frame}

\begin{frame}[fragile]
\frametitle{Работа с процессами: макросы для status}
\begin{center}
\begin{minipage}{0.95\textwidth}
\begin{minted}{c}
WIFEXITED(status) // если процесс завершился сам
WEXITSTATUS(status) // exit status
WIFSIGNALED(status) // если процесс был завершён сигналом
WTERMSIG(status) // сигнал, завершивший процесс
WCOREDUMP(status) // отложил ли процесс coredump?

\end{minted}
\end{minipage}
\end{center}
\end{frame}


% \begin{frame}[fragile]
% \frametitle{Работа с процессами: ptrace}
% \begin{center}
% \begin{minipage}{0.95\textwidth}
% \begin{minted}{c}


% // используется для ptrace
% WIFSTOPPED(status)
% WSTOPSIG(status)
% WIFCONTINUED(status)
% \end{minted}
% \end{minipage}
% \end{center}
% \end{frame}


% \begin{frame}
% \begin{itemize}
% \end{itemize}
% \end{frame}

\begin{frame}
\center\Huge{Gratias ago!}
\end{frame}


\end{document}
