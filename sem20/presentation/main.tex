\documentclass[10pt,pdf,hyperref={unicode}]{beamer}

\usepackage[normalem]{ulem}
\usepackage{qrcode}
\usepackage{array}
\usepackage[T2A]{fontenc}
\usepackage[utf8]{inputenc}
\usepackage{colortbl}
\usepackage{minted}
\usepackage{listings}
\usepackage{tcolorbox}

\setbeamertemplate{navigation symbols}{}

\usetheme{default}

\usepackage{array}
\newcolumntype{L}[1]{>{\raggedright\let\newline\\\arraybackslash\hspace{0pt}}m{#1}}
\newcolumntype{C}[1]{>{\centering\let\newline\\\arraybackslash\hspace{0pt}}m{#1}}
\newcolumntype{R}[1]{>{\raggedleft\let\newline\\\arraybackslash\hspace{0pt}}m{#1}}

\definecolor{shadecolor}{RGB}{210,210,210}
\newcommand{\asm}[1]{\mintinline{gas}{#1}}

\newcommand{\qrlinkframe}[2]{\begin{frame}{#1}
\center\qrcode[hyperlink,height=75px]{#2}
\end{frame}}

\title{Семинар 20: мультиплексирование I/O-операций}
\date{13 мая, 2020}


\begin{document}

\begin{frame}
  \titlepage
\end{frame}

\begin{frame}
\Huge{Что делать, если нужно обработать несколько клиентов одновременно?}
\end{frame}

\begin{frame}{Apache MPM}
\begin{itemize}
    \item MPM = multiprocessing module
    \item Модуль Apache, который распараллеливает обработку запросов
    \item Три режима работы: prefork/worker и event
\end{itemize}
\end{frame}

\begin{frame}{Apache MPM: prefork model}
\begin{itemize}
    \item Для каждого соединения стартует новый процесс
    \item Количество inflight-соединений = количество процессов
    \item Поддерживает какое-то количество незанятых (spare) процессов — поэтому \emph{pre-}fork
    \item Обеспечивал безопасность, однако тратил много ресурсов
\end{itemize}
\end{frame}

\begin{frame}[fragile]
\frametitle{Apache MPM: prefork model}
\begin{center}
\begin{minipage}{0.95\textwidth}
\begin{minted}{cpp}

for (;;) {
    int client_fd = accept(...);

    while (children.size() > MaxRequestsNum) {
        pid_t finished = wait(NULL);
        children.erase(finished);
    }

    pid_t pid = fork();
    if (pid == 0) {
        process_connection(client_fd);
    }

    children.push_back(pid);
}

\end{minted}
\end{minipage}
\end{center}
\end{frame}

\begin{frame}{Apache MPM: worker model}
\begin{itemize}
    \item Как сократить количество ресурсов?
    \item Будем использовать треды вместо процессов!
    \item Несколько процессов-контейнеров, чтобы один тред не мог сломать весь веб-сервер
\end{itemize}
\end{frame}

\begin{frame}{HTTP keep alive}
\begin{itemize}
    \item На каждый HTTP-запрос устанавливается отдельное TCP-соединение
    \item То есть на каждый запрос тратится дополнительно $\frac{3}{2} RTT$ на 3WH
    \item Зачем закрывать соединение? Давайте делать следующий запрос в том же!
    \item С HTTP/1.1 все соединения по-умолчанию считаются keep alive
    \item prefork/worker model + keep alive = :(
    \item Очень много idle-клиентов тратят слишком много ресурсов
    \item Можно устроить DoS почти не тратя CPU
\end{itemize}
\end{frame}

\begin{frame}{Apache MPM: event model}
\begin{itemize}
    \item Основан на worker
    \item Пусть будет один тред следит за всеми соединениями
    \item Воркеры обрабатывают только уже отдельные запросы
    \item Проблема: как в одном треде уследить за несколькими соединениями?
\end{itemize}
\end{frame}

\begin{frame}[fragile]
\frametitle{Busy waiting}
\begin{center}
\begin{minipage}{0.95\textwidth}
\begin{minted}{cpp}

for (;;) {
    for (int conn_fd : connections) {
        char unused;
        ssize_t res = recv(conn_fd, &unused, 1, MSG_PEEK);
        if (res > 0) {
            // что-то появилось в conn_fd,
            // можно отдавать в воркера
        } else if (errno == EAGAIN) {
            // пока ничего нет :(
        } else {
            // res == 0 или ошибка (но не EAGAIN),
            // можно закрыть соединение
            close(conn_fd);
        }
    }
}

\end{minted}
\end{minipage}
\end{center}
\end{frame}

\begin{frame}{I/O multiplexing}
\begin{itemize}
    \item \mintinline{c}{select}, \mintinline{c}{poll}, \mintinline{c}{epoll}
    \item Различаются в деталях, но суть одна -- они ждут пока в \emph{хотя бы в одном} из файловых дескрипторов произойдёт какое-либо событие
    \item Появились данные для чтения -- \mintinline{c}{POLLIN}
    \item Появились данные для записи -- \mintinline{c}{POLLOUT}
    \item Произошла ошибка (например, \mintinline{c}{EPIPE}) -- \mintinline{c}{POLLERR}
\end{itemize}
\end{frame}

\begin{frame}[fragile]
\frametitle{I/O multiplexing: poll}
\begin{center}
\begin{minipage}{0.95\textwidth}
\begin{minted}{cpp}
#include <poll.h>

struct pollfd {
   int   fd;         /* file descriptor */
   short events;     /* requested events */
   short revents;    /* returned events */
};

int poll(struct pollfd *fds, nfds_t nfds, int timeout);
\end{minted}
\end{minipage}
\end{center}
\end{frame}

\begin{frame}[fragile]
\frametitle{I/O multiplexing: select}
\begin{center}
\begin{minipage}{0.95\textwidth}
\begin{minted}{cpp}
#include <sys/select.h>

int select(int nfds, fd_set *readfds, fd_set *writefds,
          fd_set *exceptfds, struct timeval *timeout);

void FD_CLR(int fd, fd_set *set);
int  FD_ISSET(int fd, fd_set *set);
void FD_SET(int fd, fd_set *set);
void FD_ZERO(fd_set *set);

\end{minted}
\end{minipage}
\end{center}
\end{frame}

\begin{frame}{I/O multiplexing: poll \& select}
\begin{itemize}
    \item Самый старые интерфейсы: \mintinline{c}{poll} появился ещё в прошлом тысячелетии (2.1.x), а \mintinline{c}{select} в 00-ых (2.6.x)
    \item Почти все POSIX-совместимые ОС реализуют select, самый переносимый вариант
    \item \mintinline{c}{fd_set} ограничен 128 байтами, $\Rightarrow$ нельзя использовать для файловых дескрипторов больше 1024
    \item Если файловых дескрипторов слишком много, то происходит много копирований (как в userspace, так и при вызове \mintinline{c}{poll})
\end{itemize}
\end{frame}

\begin{frame}[fragile]
\frametitle{I/O multiplexing: epoll}
\begin{center}
\begin{minipage}{0.95\textwidth}
\begin{minted}{cpp}
#include <sys/epoll.h>

struct epoll_event {
    uint32_t     events;
    union {
        void    *ptr;
        int      fd;
        uint32_t u32;
        uint64_t u64;
    } data;
};

int epoll_create1(int flags);

int epoll_ctl(int epfd, int op, int fd,
                      struct epoll_event *event);

int epoll_wait(int epfd, struct epoll_event *events,
                      int maxevents, int timeout);

\end{minted}
\end{minipage}
\end{center}
\end{frame}

\begin{frame}{I/O multiplexing: epoll}
\begin{itemize}
    \item Немного меняется семантика: в epoll нужно регистрировать (\mintinline{c}{EPOLL_CTL_ADD}) файловые дескрипторы, а затем \mintinline{c}{epoll_wait} возвращает события, которые произошли с ними
    \item С каждым файловым дескриптором можно ассоциировать какие-то данные (например, указатель на класс, которые отвечает за соединение)
    \item Файловые дескрипторы можно подписывать на разные события: \mintinline{c}{EPOLLIN}, \mintinline{c}{EPOLLOUT}, \mintinline{c}{EPOLLHUP}, ...
    \item Под файловыми дескрипторами могут скрываться сокеты, пайпы, а также специальные механизмы, вроде \mintinline{c}{eventfd} или \mintinline{c}{timerfd}
    \item \textbf{ВАЖНО}: epoll на регулярных файлах всегда будет сразу выдавать готовые события
\end{itemize}
\end{frame}


\begin{frame}{epoll: edge-triggered vs level-triggered}
\begin{itemize}
    \item Разные способы оповещения о событиях для конкретных файловых дескрипторов
    \item Level-triggered: оповещения будут приходить пока вы полностью не исчерпаете событие (например, не прочитаете весь read buffer)
    \item Edge-triggered: оповещение будет приходить \textbf{только один раз}, как только вы его получите через \mintinline{c}{epoll_wait}, epoll о нём забудет
\end{itemize}
\end{frame}

\begin{frame}{Coroutines}
\begin{itemize}
    \item Почему бы внутри программы не попробовать самим управлять потоками?
    \item Корутины (или green threads, или greenlets) -- треды, которые управляются программой
    \item Обычно используется cooperative multitasking
    \item Корутины работают, пока не дойдут до блокирующей I/O-операции, затем передают управление другим
    \item Как только running корутин не осталось, корутиновый движок зависает на epoll/poll/select
    \item Затем пробуждаются корутины и итерация повторяется
\end{itemize}
\end{frame}

\begin{frame}
\center\Huge{Спасибо!}
\end{frame}


\end{document}
